\documentclass{article} %article 文档
\usepackage{ctex}  %使用宏包(为了能够显示汉字)
\usepackage[hidelinks]{hyperref}
\usepackage{color}
\title{高等数学笔记}  %文章标题
\author{}   %作者的名称
\date{}       %日期
% 设置页面的环境,a4纸张大小,左右上下边距信息
\usepackage[a4paper,left=10mm,right=10mm,top=15mm,bottom=15mm]{geometry}  
\begin{document} %正文部分
\maketitle          %添加这一句才能够显示标题等信息

%目录页
\tableofcontents
\thispagestyle{empty}
\clearpage
\setcounter{page}{1}



\section{向量代数与空间解析几何}
\subsection{向量及其线性运算}
\subsubsection{向量的概念}
\begin{description}
    \item[向量]{既有大小,又有方向的量。记作$\overrightarrow{a}$} 
    \item[模]{向量的大小。记作$|\overrightarrow{a}|$} 
    \item[$\overrightarrow{a}$与$\overrightarrow{b}$的夹角]{记作$\widehat{(\overrightarrow{a},\overrightarrow{b})}$,$\widehat{(\overrightarrow{a},\overrightarrow{b})}\in[0,\pi]$} 
\end{description}

若$\widehat{(\overrightarrow{a},\overrightarrow{b})}=0$或$\pi$,则$\overrightarrow{a}//\overrightarrow{b}$\par
若$\widehat{(\overrightarrow{a},\overrightarrow{b})}=\frac{\pi}{2}$,则$\overrightarrow{a}\bot \overrightarrow{b}$\par

\subsubsection{向量的线性运算}
\begin{enumerate}
\item \textbf{向量的加减法}\par
\textbf{运算规律:}\par
\begin{enumerate}
    \item 交换律:$\overrightarrow{a}+\overrightarrow{b}=\overrightarrow{b}+\overrightarrow{a}$
    \item 结合律:$(\overrightarrow{a}+\overrightarrow{b})+\overrightarrow{c}
    =\overrightarrow{a}+(\overrightarrow{b}+\overrightarrow{c})$
\end{enumerate}
\textbf{三角不等式:}\par
$|\overrightarrow{a}\pm \overrightarrow{b}|\leq 
|\overrightarrow{a}|+|\overrightarrow{b}|
$

\item \textbf{向量的数乘}\par
\textbf{运算规律:}\par
\begin{enumerate}
    \item 结合律:$\lambda(\mu\overrightarrow{a})
    =\mu(\lambda\overrightarrow{a})
    =(\lambda\mu)\overrightarrow{a}
    $
    \item 分配律:\begin{tabular}{l}
        $(\lambda+\mu) \overrightarrow{a}=\lambda\overrightarrow{a}+\mu \overrightarrow{a}$\\
        $\lambda(\overrightarrow{a}+\overrightarrow{b})=\lambda\overrightarrow{a}+\lambda \overrightarrow{a}$
    \end{tabular}
\end{enumerate}
\end{enumerate}

$\overrightarrow{a}//\overrightarrow{b}\Leftrightarrow \overrightarrow{a}=\lambda\overrightarrow{b}$

\subsubsection{空间直角坐标系}

\subsubsection{利用坐标做向量的线性运算}
\begin{enumerate}
    \item 定比分点\\$M$位于$A(x_1,y_1,z_1),B(x_2,y_2,z_2)$中间,且$\overrightarrow{AM}=\lambda\overrightarrow{MB}$\\
    则$M(\frac{x_1+\lambda x_2}{1+\lambda},\frac{y_1+\lambda y_2}{1+\lambda},\frac{z_1+\lambda z_2}{1+\lambda})$
\end{enumerate}

\subsubsection{向量的模、方向角、投影}
\begin{enumerate}
    \item \textbf{向量的模与两点间的距离公式}\\
    $|\overrightarrow{a}|=\sqrt{x^2+y^2+z^2}$\\
    $|AB|=|\overrightarrow{AB}|=\sqrt{(x_1-x_2)^2+(y_1-y_2)^2+(z_1-z_2)^2}$

    \item \textbf{方向角与方向余弦}
    \begin{description}
        \item[方向角]{$\overrightarrow{a}$与坐标轴$x,y,z$分别所成的角$\alpha,\beta,\gamma$}
        \item[方向余弦]{$\cos\alpha,\cos\beta,\cos\gamma$} 
    \end{description}
    $\displaystyle(\cos\alpha,\cos\beta,\cos\gamma)=(\frac{x}{|\overrightarrow{a}|},\frac{y}{|\overrightarrow{a}|},\frac{z}{|\overrightarrow{a}|})$\\
    $\cos^2\alpha+\cos^2\beta+\cos^2\gamma=1$

    \item \textbf{向量在轴上的投影}\\
    $\displaystyle a_x=Prj_{\overrightarrow{i}}\overrightarrow{a}=(\overrightarrow{a})_x,a_y=Prj_{\overrightarrow{j}}\overrightarrow{a}=(\overrightarrow{a})_y,a_z=Prj_{\overrightarrow{k}}\overrightarrow{a}=(\overrightarrow{a})_z$
    \begin{description}
        \item [性质1]{$(\overrightarrow{a})_u=|\overrightarrow{a}|cos\phi$,其中$\phi$为$\overrightarrow{a}$与坐标轴$u$的夹角}
        \item [性质2]{$(\overrightarrow{a}+\overrightarrow{b})_u=(\overrightarrow{a})_u+(\overrightarrow{b})_u$}
        \item [性质3]{$(\lambda\overrightarrow{a})_u=\lambda(\overrightarrow{a})_u$}
    \end{description}
\end{enumerate}

\subsection{数量积、向量积、*混合积}
\subsubsection{两向量的数量积}
\begin{enumerate}
    \item \textbf{计算}\\
    $\overrightarrow{a}\cdot\overrightarrow{b}=|\overrightarrow{a}||\overrightarrow{b}|\cos\theta\\
    =|\overrightarrow{a}||\overrightarrow{b}|\cos\widehat{(\overrightarrow{a},\overrightarrow{b})}\\
    =|\overrightarrow{a}|Prj_{\overrightarrow{a}}\overrightarrow{b}\\
    =a_xb_x+a_yb_y+a_zb_z$

    \item \textbf{性质}\begin{enumerate}
        \item $\overrightarrow{a}\cdot\overrightarrow{b}=|\overrightarrow{a}|^2$
        \item $\overrightarrow{a}\bot \overrightarrow{b}\Leftrightarrow\overrightarrow{a}\cdot\overrightarrow{b}=0$
    \end{enumerate}

    \item \textbf{运算规律}\begin{enumerate}
        \item 交换律 $\overrightarrow{a}\cdot\overrightarrow{b}=\overrightarrow{b}\cdot\overrightarrow{a}$
        \item 分配律 $(\overrightarrow{a}+\overrightarrow{b})\cdot\overrightarrow{c}=\overrightarrow{a}\cdot\overrightarrow{c}+\overrightarrow{b}\cdot\overrightarrow{c}$
        \item 结合律 $(\lambda\overrightarrow{a})\cdot\overrightarrow{b}=\lambda(\overrightarrow{a}\cdot\overrightarrow{b})$
    \end{enumerate}
    \item \textbf{向量夹角}\\$\cos\theta=\frac{\overrightarrow{a}\cdot\overrightarrow{b}}{|\overrightarrow{a}||\overrightarrow{b}|}$
\end{enumerate}
\subsubsection{两向量的向量积}
\begin{enumerate}
    \item \textbf{计算}\\
    $|\overrightarrow{c}|=|\overrightarrow{a}||\overrightarrow{b}|\sin\theta$,$\overrightarrow{c}$的方向由右手规则(握拳时四指旋转方向夹角小于$\pi$时大拇指指向的方向)确定\\
    $\overrightarrow{c}=\overrightarrow{a}\times \overrightarrow{b}
    =\left|
        \begin{array}{ccc}
            \overrightarrow{i} & \overrightarrow{j} & \overrightarrow{k} \\
            a_x & a_y & a_z \\
            b_x & b_y & b_z \\
        \end{array}
    \right|
    $\\

    \item \textbf{性质}\begin{enumerate}
        \item $\overrightarrow{a}\times\overrightarrow{b}=0$
        \item $\overrightarrow{a}// \overrightarrow{b}\Leftrightarrow\overrightarrow{a}\times\overrightarrow{b}=0$
    \end{enumerate}

    \item \textbf{运算规律}\begin{enumerate}
        \item $\overrightarrow{a}\times\overrightarrow{b}=-\overrightarrow{b}\times\overrightarrow{a}$
        \item 分配律 $(\overrightarrow{a}+\overrightarrow{b})\times\overrightarrow{c}=\overrightarrow{a}\times\overrightarrow{c}+\overrightarrow{b}\times\overrightarrow{c}$
        \item 结合律 $(\lambda\overrightarrow{a})\times\overrightarrow{b}=\overrightarrow{a}\times(\lambda\overrightarrow{b})=\lambda(\overrightarrow{a}\times\overrightarrow{b})$
    \end{enumerate}

\end{enumerate}
\subsubsection{*向量的混合积}
$(\overrightarrow{a}\times\overrightarrow{b})\cdot\overrightarrow{c}=
\left|\begin{array}{ccc}
    a_x & a_y & a_z \\
    b_x & b_y & b_z \\
    c_x & c_y & c_z \\
\end{array}\right|
$

\textbf{性质:}$(\overrightarrow{a}\times\overrightarrow{b})\cdot\overrightarrow{c}=0 \Leftrightarrow
\overrightarrow{a},\overrightarrow{b},\overrightarrow{c}$共面

\subsection{平面及其方程}
\subsubsection{曲面方程与空间曲线方程的概念}
\begin{enumerate}
    \item \textbf{曲面方程}\\若曲面$S$与方程$F(x,y,z)=0$有如下关系:
    \begin{enumerate}
        \item 曲面$S$上任意一点都满足方程
        \item 曲面$S$外任意一点都不满足方程
    \end{enumerate}

    那么方程$F(x,y,z)=0$就叫做\textcolor{red}{曲面$S$的方程}\\
    曲面$S$就叫做\textcolor{red}{方程$F(x,y,z)=0$的图形}

    \item \textbf{曲线方程}\\
    空间曲线$C$的方程
    $\left\{\begin{array}{l}
        F(x,y,z)=0\\
        G(x,y,z)=0
    \end{array}
    \right.
    $
\end{enumerate}

\subsubsection{平面的方程}
\begin{description}
    \item [平面的点法式方程]{\textcolor{red}{$A(x-x_0)+B(y-y0)+C(z-z_0)=0$},已知平面$\Pi$上一点$M_0(x_0,y_0,z_0)$和一个法向量$\overrightarrow{n}=(A,B,C)$}
    \item [平面的三点式方程]{\textcolor{red}{$\left|\begin{array}{ccc}
        x-x_0 & y-y_0 & z-z_0 \\
        x_1-x_0 & y_1-y_0 & z_1-z_0 \\
        x_2-x_0 & y_2-y_0 & z_2-z_0 \\
        \end{array}\right|$},已知平面$\Pi$上3点$M_0(x_0,y_0,z_0),M_1(x_1,y_1,z_1),M_2(x_2,y_2,z_2)$}
    \item [平面的截距式方程]{\textcolor{red}{$\displaystyle\frac{x}{a}+\frac{y}{b}+\frac{z}{c}=1$},已知平面$\Pi$在$x$轴$y$轴$z$轴上的截距分别为$a,b,c$}
    \item [平面的一般式方程]{\textcolor{red}{$Ax+By+Cz+D=0$}}
\end{description}


\subsubsection{两平面的夹角}
定义:两平面的法线向量的夹角$\theta$(通常指锐角或直角)\par
平面$\Pi_1,\Pi_2$的法向量分别为$\overrightarrow{n_1}=(A_1,B_1,C_1),\overrightarrow{n_2}=(A_2,B_2,C_2)$\par
则$\Pi_1,\Pi_2$之间的夹角的余弦值为\textcolor{red}{$\displaystyle\cos\theta={|A_1A_2+B_1B_2+C_1C_2|\over\sqrt{A_1^2+B_1^2+C_1^2}\sqrt{A_2^2+B_2^2+C_2^2}}$}\par

\textbf{性质}
\begin{enumerate}
    \item $\Pi_1\bot\Pi_2 \Leftrightarrow A_1A_2+B_1B_2+C_1C_2=0$
    \item $\displaystyle\Pi_1//\Pi_2 \Leftrightarrow\frac{A_1}{A_2}=\frac{B_1}{B_2}=\frac{C_1}{C_2}$
\end{enumerate}

\subsubsection{点到平面的距离}
已知点$M(x_0,y_0,z_0)$、平面方程$Ax+By+Cz+D=0$\par
则点到平面度的距离\textcolor{red}{$\displaystyle d={|Ax_0+By_0+Cz_0+D| \over \sqrt{A^2+B^2+C^2}}$}

\subsection{空间直线及其方程}
\subsubsection{空间直线的方程}
\begin{description}
    \item[一般式方程]{\textcolor{red}{
    $\left\{\begin{array}{l}
        A_1x+B_1y+C_1z+D_1=0 \\
        A_2x+B_2y+C_2z+D_2=0 \\
    \end{array}\right.$}
    }

    \item[对称式方程]{\textcolor{red}{
    $\displaystyle\frac{x-x_0}{m}=\frac{y-y_0}{n}=\frac{z-z_0}{p}$}
    ,已知方向向量$\overrightarrow{s}=(m,n,p)$,直线上一点$M_0(x_0,y_0,z_0)$
    }

    \item[参数式方程]{\textcolor{red}{
    $\left\{\begin{array}{l}
        x=x_0+mt \\
        y=y_0+nt \\
        z=z_0+pt \\
    \end{array}\right.$}
    ,其中$\displaystyle\frac{x-x_0}{m}=\frac{y-y_0}{n}=\frac{z-z_0}{p}=t$
    }

\end{description}

\subsubsection{两直线的夹角}
定义:两直线的方向向量的夹角$\theta$(通常指锐角或直角)\par
直线$L_1,L_2$的方向向量分别为$\overrightarrow{s_1}=(A_1,B_1,C_1),\overrightarrow{n_2}=(A_2,B_2,C_2)$\par
则$L_1,L_2$之间的夹角的余弦值为\textcolor{red}{$\displaystyle\cos\theta={|A_1A_2+B_1B_2+C_1C_2|\over\sqrt{A_1^2+B_1^2+C_1^2}\sqrt{A_2^2+B_2^2+C_2^2}}$}\par

\subsubsection{直线与平面的夹角}
定义:当直线与平面不垂直时,直线和它在平面上的投影直线的夹角$\theta(0\leq \theta<\frac{\pi}{2})$\par
直线$L$的方向向量为$\overrightarrow{s}=(a,b,c)$,平面$\Pi$的法向量为$\overrightarrow{n}=(A,B,C)$\par
则$L,\Pi$之间的夹角的正弦值为\textcolor{red}{$\displaystyle\sin\theta={|Aa+Bb+Cc|\over\sqrt{A^2+B^2+C^2}\sqrt{a^2+b^2+c^2}}$}\par

\subsection{曲面及其方程}
\subsubsection{旋转曲面}
\begin{description}
    \item[旋转曲面] {一条平面曲线绕其平面上的一条直线旋转一周所成的曲面}
    \item[母线] {旋转的曲线}
    \item[轴] {定直线}
\end{description}

\subsubsection{柱面}
\begin{description}
    \item[柱面] {直线$L$沿定曲线$C$平移形成的轨迹}
    \item[母线] {定曲线$C$}
    \item[准线] {动直线$L$}
\end{description}

\subsubsection{二次曲面}
\begin{enumerate}
    \item \textbf{椭圆锥面}\hspace{2mm}
    \textcolor{red}{$\displaystyle\frac{x^2}{a^2}+\frac{y^2}{b^2}=z^2$}

    \item \textbf{椭球面}\hspace{2mm}
    \textcolor{red}{$\displaystyle\frac{x^2}{a^2}+\frac{y^2}{b^2}+\frac{z^2}{c^2}=1$}

    \item \textbf{单叶双曲面}\hspace{2mm}
    \textcolor{red}{$\displaystyle\frac{x^2}{a^2}+\frac{y^2}{b^2}-\frac{z^2}{c^2}=1$}

    \item \textbf{双叶双曲面}\hspace{2mm}
    \textcolor{red}{$\displaystyle\frac{x^2}{a^2}-\frac{y^2}{b^2}-\frac{z^2}{c^2}=1$}

    \item \textbf{椭圆抛物面}\hspace{2mm}
    \textcolor{red}{$\displaystyle\frac{x^2}{a^2}+\frac{y^2}{b^2}=z$}

    \item \textbf{双曲抛物面}\hspace{2mm}
    \textcolor{red}{$\displaystyle\frac{x^2}{a^2}-\frac{y^2}{b^2}=z$}

    \item \textbf{椭圆柱面}
    \item \textbf{双曲柱面}
    \item \textbf{抛物柱面}
\end{enumerate}

\subsection{空间曲线及其方程}
\subsubsection{空间曲线的方程}
\begin{itemize}
    \item \textbf{一般方程}\hspace{2mm}\textcolor{red}{
        $\left\{\begin{array}{l}
            F(x,y,z)=0 \\
            G(x,y,z)=0 \\
        \end{array}\right.$
    }

    \item \textbf{参数方程}\hspace{2mm}\textcolor{red}{
        $\left\{\begin{array}{l}
            x=x(t) \\
            y=y(t) \\
            z=z(t) \\
        \end{array}\right.$
    }
\end{itemize}

\subsubsection{空间曲线在坐标面上的投影}
求法:将不需要的维度消掉,得到柱面公式,再令该维度为0即可得到
\begin{itemize}
    \item \textbf{在$xOy$平面上的投影}\hspace{2mm}\textcolor{red}{
        $\left\{\begin{array}{l}
            H(x,y)=0 \\
            z=0 \\
        \end{array}\right.$
    }

    \item \textbf{在$xOz$平面上的投影}\hspace{2mm}\textcolor{red}{
        $\left\{\begin{array}{l}
            H(x,z)=0 \\
            y=0 \\
        \end{array}\right.$
    }

    \item \textbf{在$yOz$平面上的投影}\hspace{2mm}\textcolor{red}{
        $\left\{\begin{array}{l}
            H(y,z)=0 \\
            x=0 \\
        \end{array}\right.$
    }
\end{itemize}


\end{document}

% \hspace{1cm}