\documentclass{article} %article 文档
\usepackage{ctex}  %使用宏包(为了能够显示汉字)
\usepackage[hidelinks]{hyperref}
\usepackage{color}
\usepackage{amsmath}  %公式符号包

\title{高等数学笔记}  %文章标题
\author{}   %作者的名称
\date{}       %日期
% 设置页面的环境,a4纸张大小,左右上下边距信息
\usepackage[a4paper,left=10mm,right=10mm,top=15mm,bottom=15mm]{geometry}  
\begin{document} %正文部分
\maketitle          %添加这一句才能够显示标题等信息

%目录页
\tableofcontents
\thispagestyle{empty}
\clearpage
\setcounter{page}{1}





%正文
\section{重积分}
\subsection{二重积分的概念与性质}
\subsubsection{二重积分的概念}
$$ \iint_{D} f(x,y)d\sigma =\iint_{D} f(x,y)dxdy $$
\begin{enumerate}
    \item 被积函数$f(x,y)$
    \item 被积表达式$f(x,y)d\sigma$
    \item 面积元素$d\sigma$
    \item 积分变量$x,y$
    \item 积分区域$D$
\end{enumerate}


\subsubsection{二重积分的性质}
\begin{enumerate}
    \item $\displaystyle \iint_{D} [\alpha f(x,y)+\beta g(x,y)]d\sigma =\alpha \iint_{D} f(x,y)d\sigma + \beta \iint_{D} g(x,y)d\sigma $
    \item $\displaystyle \iint_{D} \alpha f(x,y)d\sigma =\iint_{D_1} f(x,y)d\sigma + \iint_{D_2} f(x,y)d\sigma $
    \item $\displaystyle \sigma=\iint_D 1\cdot d\sigma=\iint_D d\sigma$,$\sigma$为$D$的面积
    \item 若在$D$上$f(x,y)\leq g(x,y)$,则$\displaystyle\iint_D f(x,y)d\sigma \leq \iint_D g(x,y)d\sigma$\\
        $\displaystyle \left|\iint_D f(x,y)d\sigma \right|\leq \iint_D |f(x,y)|d\sigma $
    \item $\displaystyle m\sigma\leq\iint_D f(x,y)d\sigma\leq M\sigma$,其中$m,M$分别为$f(x,y)$在$D$上的最小值和最大值
    \item 二重积分中值定理$\displaystyle \iint_D f(x,y)d\sigma=f(\xi,\eta)\sigma$,其中$(\xi,\eta)\in D$
\end{enumerate}

\subsection{二重积分的计算法}
\subsubsection{利用直角坐标系计算二重积分}
$$\iint_D f(x,y)d\sigma = \int_a^b\left[\int_{y_1(x)}^{y_2(x)}f(x,y)dy\right]dx=\int_a^b dx\int_{y_1(x)}^{y_2(x)}f(x,y)dy$$\par
简单来说就是把积分拆成2次计算,\par
先把$x$看成常数计算$y$从$y_1(x)$到$y_2(x)$的定积分,\par
然后再把结果对$x$计算$[a,b]$上的定积分\par

\textcolor{red}{关键:确定积分限}

\subsubsection{利用极坐标系计算二重积分}
$$\begin{array}{rl}
\displaystyle\iint_D f(\rho\cos\theta,\rho\sin\theta)\rho d\rho d\theta 
&\displaystyle = \int_{\alpha}^{\beta}\left[\int_{\phi_1(\theta)}^{\phi_2(\theta)}f(\rho\cos\theta,\rho\sin\theta)\rho d\rho \right]d\theta 
\\ &\displaystyle = \int_{\alpha}^{\beta}d\theta \int_{\phi_1(\theta)}^{\phi_2(\theta)}f(\rho\cos\theta,\rho\sin\theta)\rho d\rho
\end{array}$$

\textbf{重要结论:}\textcolor{red}{$$\displaystyle
\int_0^{\frac{\pi}{2}}\sin^n\theta\mathrm{d}\theta=\int_0^{\frac{\pi}{2}}\cos^n\theta\mathrm{d}\theta=\left\{
    \begin{array}{ll}
        \displaystyle \frac{n-1}{n}\cdot\frac{n-3}{n-2}\cdot...\cdot\frac{3}{4}\cdot\frac{1}{2}\cdot\frac{\pi}{2} & \mbox{,n为正偶数}\vspace{2mm}\\
        \displaystyle \frac{n-1}{n}\cdot\frac{n-3}{n-2}\cdot...\cdot\frac{4}{5}\cdot\frac{2}{3} & \mbox{,n为大于1的正奇数}\\
    \end{array}
\right.
$$}
\subsubsection{*二重积分的换元法}

\subsection{三重积分}
\subsubsection{三重积分的概念}
$$\iiint_\Omega f(x,y,z)dv=\iiint_\Omega f(x,y,z)dxdydz$$

\subsubsection{三重积分的计算}

简单来说就是把积分拆成3次计算\par
\begin{enumerate}
    \item 直角坐标系\\
        $\displaystyle \iiint_\Omega f(x,y,z)dv=\int_{a}^{b}dx\int_{y_1(x)}^{y_2(x)}dy\int_{z_1(x,y)}^{z_2(x,y)}f(x,y,z)dz$
    \item 柱面坐标系$\left\{\begin{array}{l}
            x=\rho\cos \theta\\
            y=\rho\sin \theta\\
            z=z
        \end{array}\right.$\\
        $\displaystyle \iiint_\Omega f(x,y,z)dv=\iiint_{\Omega}F(\rho,\theta,z)\rho d\rho d\theta dz$
    \item *球面坐标系$\left\{\begin{array}{l}
            x=r\cos\theta\sin\phi\\
            y=r\sin\theta\sin\phi\\
            z=r\cos\phi
        \end{array}\right.$\\
        
\end{enumerate}

\subsection{重积分的应用}
\subsubsection{曲面面积}
面积元素$dA=\sqrt{1+f_x^2(x,y)+f_y^2(x,y)}d\sigma$\par
曲面面积$\begin{array}{rl}
    A &\displaystyle =\iint_{D_{xy}} \sqrt{1+f_x^2(x,y)+f_y^2(x,y)}d\sigma
\end{array}$\par

*用曲面参数方程求曲面面积

\subsubsection{质心}
平面薄板的质心\par
$\displaystyle\overline{x}=\frac{M_y}{M}{\displaystyle\iint_D x\mu(x,y)d\sigma\over\displaystyle\iint_D \mu(x,y)d\sigma}$\par
$\displaystyle\overline{y}=\frac{M_x}{M}{\displaystyle\iint_D y\mu(x,y)d\sigma\over\displaystyle\iint_D \mu(x,y)d\sigma}$\par

\subsubsection{转动惯量}
$dI_x=y^2\mu(x,y)d\sigma,dI_y=x^2\mu(x,y)d\sigma$\par
$\displaystyle I_x=\iint_D y^2\mu(x,y)d\sigma$\par
$\displaystyle I_y=\iint_D x^2\mu(x,y)d\sigma$\par

\subsubsection{引力}
一空间物体对某一质点的引力\par
$\begin{array}{rl}
    \vec{F} & =(\vec{F}_x,\vec{F}_y,\vec{F}_z)\\
      & \displaystyle =(\iiint_\Omega{G\rho(x,y,z)(x-x_0) \over r^3}dv,\iiint_\Omega{G\rho(x,y,z)(y-y_0) \over r^3}dv,\iiint_\Omega{G\rho(x,y,z)(z-z_0) \over r^3}dv )
\end{array}$

\subsection{*含参变量的积分}



\end{document}