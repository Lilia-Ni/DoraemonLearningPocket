\documentclass{article} %article 文档
\usepackage{ctex}  %使用宏包(为了能够显示汉字)
\usepackage[hidelinks]{hyperref}
\usepackage{color}
\usepackage{amsmath}  %公式符号包

\title{高等数学笔记}  %文章标题
\author{}   %作者的名称
\date{}       %日期
% 设置页面的环境,a4纸张大小,左右上下边距信息
\usepackage[a4paper,left=10mm,right=10mm,top=15mm,bottom=15mm]{geometry}  
\begin{document} %正文部分
\maketitle          %添加这一句才能够显示标题等信息

%目录页
\tableofcontents
\thispagestyle{empty}
\clearpage
\setcounter{page}{1}





%正文
\section{曲线积分与曲面积分}
\subsection{对弧长的曲线积分}
\subsubsection{对弧长的曲线积分的概念与性质}
第一类曲线积分,记为$\displaystyle \int_Lf(x,y)ds$,$\displaystyle\int_\Gamma f(x,y,z)ds$\par
若曲线是闭合的,记为$\displaystyle \oint_Lf(x,y)ds$\par
性质
\begin{enumerate}
    \item $\displaystyle \int_L [\alpha f(x,y)+\beta g(x,y)]ds= \alpha\int_L f(x,y)ds+\beta\int_L g(x,y)ds$
    \item $\displaystyle \int_Lf(x,y)ds=\int_{L_1}f(x,y)ds+\int_{L_2}f(x,y)ds$
    \item 若$f(x,y)\leq g(x,y)$,则$\displaystyle\int_L f(x,y)ds\leq\int_Lg(x,y)ds$\\
        $\displaystyle\left|\int_Lf(x,y)ds\right|=\int_L|f(x,y)|ds$
\end{enumerate}

\subsubsection{对弧长的曲线积分的计算法}
若曲线$L$的参数方程为$\left\{\begin{array}{ll}
x=g(t)\\
y=h(t)
\end{array}\right.$\par
则曲线积分$\displaystyle \int_L f(x,y)ds=\int_{\alpha}^{\beta}f[g(t),h(t)]\sqrt{g'^2(t)+h'^2(t)}dt,(\alpha<\beta)$

\subsection{对坐标的曲线积分}
\subsubsection{对坐标的曲线积分的概念与性质}
第二类积分,对函数$\vec{F}(x,y)=P(x,y)\vec{i}+Q(x,y)\vec{j}$\par
$P(x,y)$在$L$上对$x$坐标的曲线积分为$\displaystyle \int_L P(x,y)dx$(L在x方向上的积分)\par
$Q(x,y)$在$L$上对$y$坐标的曲线积分为$\displaystyle \int_L Q(x,y)dy$(L在y方向上的积分)\par
对空间函数同理\par
\vspace{5mm}
本积分主要应用于向量函数的积分\par
$\displaystyle \int_L\vec{F}(x,y)d\vec{r}=\int_{L}[P(x,y)dx+Q(x,y)dy]$
其中$d\vec{r}=dx\vec{i}+d\vec{j}$\par

\vspace{5mm}
\textbf{性质}
\begin{enumerate}
    \item $\alpha,\beta$为常数,则
        $$\displaystyle \int_L[\alpha\vec{F_1}(x,y)d\vec{r}+\beta\vec{F_2}(x,y)d\vec{r}]=
        \alpha\int_{L}\vec{F_1}(x,y)d\vec{r}+\beta\int_{L}\vec{F_2}(x,y)d\vec{r}$$

    \item 若有向曲线弧$L$可分成两段光滑的有向曲线弧$L_1,L_2$,则
        $$\int_L\vec{F}(x,y)d\vec{r}=\int_{L_1}\vec{F}(x,y)d\vec{r}+\int_{L_2}\vec{F}(x,y)d\vec{r}$$
    
    \item $L^-$是$L$的反向曲线弧,则
        $$\int_{L^-}\vec{F}(x,y)d\vec{r}=-\int_{L}\vec{F}(x,y)d\vec{r}$$
        \textcolor{red}{注:由此可知,在对坐标曲线积分时,我们必须注意积分弧段的方向}
\end{enumerate}

\subsubsection{对坐标的曲线积分的计算法}
$L$的参数方程为$\left\{\begin{array}{ll}
    x=g(t)\\
    y=h(t)
\end{array}\right.$\par
$\displaystyle \int_{L}[P(x,y)dx+Q(x,y)dy]=\int_{\alpha}^{\beta}\{P[g(t),h(t)]g'(t)+Q[g(t),h(t)]h'(t)\}dt$
\par
\vspace{3mm}
计算时,只要把$x,y,dx,dy$依次替换为$g(t),h(t),g'(t)dt,h'(t)dt$然后从起点到终点积分即可。\par
\textcolor{red}{注意:下限$\alpha$ 对应于 $L$ 的起点,上限$\beta$ 对应于$L$的终点,$\alpha$不一定小于$\beta$}
\par
\vspace{5mm}
空间曲线计算同理

\subsubsection{两类曲线积分之间的联系}
$\displaystyle \int_L Pdx+Qdy=\int_L(P\cos\alpha+Q\cos\beta)ds$
\par
\vspace{5mm}
% 用向量的形式表示为
% $$


\subsection{格林公式及其应用}
\subsubsection{格林公式}
在平面$D$上的二重积分可以通过沿闭区域$D$的边界曲线$L$上的曲线积分来表示。
\begin{itemize}
    \item 单连通区域:无洞的区域
    \item 复连通区域:有洞的区域
\end{itemize}
\par
格林公式:$\displaystyle \iint_D \left(\frac{\partial Q}{\partial x}-\frac{\partial P}{\partial y}\right)dxdy=\oint_L Pdx+Qdy$
\par


















\section{}
\subsection{}
\subsubsection{}

\end{document}